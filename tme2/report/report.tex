
\documentclass{article}

%% Packages for French writing
\usepackage[french]{babel}
\usepackage[utf8]{inputenc}
\usepackage[T1]{fontenc}
\usepackage{layout}

%% Packages for Math symbols
\usepackage[fleqn]{amsmath}
\usepackage{amssymb}
\usepackage{mathrsfs}

%% Packages for figures insertion
\usepackage{graphicx}
\usepackage{wrapfig}
\usepackage{framed}
\usepackage{float}

%% Package for document margin editing
\usepackage[top=2cm, bottom=2cm, left=2cm, right=2cm]{geometry}

%% Package for source code insertion
\usepackage{listings}
\usepackage{xcolor}
\definecolor{grey}{rgb}{0.97, 0.97, 0.97}
\definecolor{darkred}{rgb}{0.42, 0, 0}
\definecolor{darkblue}{rgb}{0, 0, 0.42}
\definecolor{darkgrey}{rgb}{0.22, 0.22, 0.82}
\definecolor{green}{HTML}{088A08}
\lstset{
  basicstyle=\small\sffamily\footnotesize,
  captionpos=b,
  numbers=left,
  numberstyle=\tiny,
  tabsize=4,
  frame=trBL,
  backgroundcolor=\color{grey},
  commentstyle=\color{green},
  keywordstyle=\color{darkblue}\bf,
  identifierstyle=\color{darkgrey},
  stringstyle=\color{darkred}
}

\setlength\parindent{0pt}
\setlength\parskip{3pt}
\title{ARCHI2 - Compte-rendu du TME2}
\author{Nicolas Phan}
\date{pour le 23 Février 2018}
\begin{document}
\pagestyle{headings}
\maketitle
\tableofcontents
\newpage

%===============================================================================
%=========================  Introduction  ======================================
%===============================================================================


\section{Modélisation de l'architecture matérielle}

\subsection{Question C1}

Caractéristique des caches de données et d'instructions :
\begin{itemize}
 \item \textbf{taille :} 1024 octets
 \item \textbf{mapping :} direct mapping, pas d'associativité
 \item \textbf{taille ligne :} 16 octets
 \item \textbf{profondeur de TEP :} 8 octets
\end{itemize}

La taille d'une ligne de cache est de 16 octets, autrement dit 4 mots, d'où:
$\texttt{icache\_words} = 4$.

Le nombre de lignes de cache est :
\[ \text{nombre de lignes} =
   \frac{\text{taille totale du cache}}
	{\text{taille d'une ligne}}
    = \frac{1024}{16} = 64
\]
Puis comme le mapping est direct :
\[ \text{nombre d'ensembles} = \text{nombre de lignes} = 64 \]

D'où $\texttt{icache\_sets} = 64$.

Le cache est à correspondance directe, il n'y a donc qu'un niveau
d'associativité.
D'où $\texttt{icache\_ways} = 1$.

\subsection{Question C2}

Au démarrage de la machine, la RAM est vide. Pour que la machine démarre,
il faut bien qu'un premier programme s'exécute et charge l'OS de la ROM vers
la RAM notamment.
Ce programme, le code de boot, doit être nécéssairement dans une mémoire
contenant des données au moment du démarrage, autrement dit dans la ROM.

\subsection{Question C3}

Contrairement aux autres segments, le segment TTY réfère à des registres qui
peuvent être modifiés par un composant autre que le processeur, sans que le
processeur le sache.
Cacher ces registres entrainerait donc des incohérences mémoire car une
donnée cached deviendait incohérente sans que le processeur le sache.

\subsection{Question C4}

Les segments kcode, kunc, kdata contiennent des données appartenant au noyau,
elle sont donc protégées, de plus le segment tty contient des registres
permettant de communiquer avec le périphérique TTY et seul l'OS a le droit de
communiquer directement avec les périphériques (les programmes user doivent
passer par lui s'ils veulent communiquer avec un périphérique) donc le segment
TTY est protégé aussi.

La protection de ces segments est réalisée en les plaçant dans la moitié haute
de l'espace d'adressage. Ce découpage de l'espace d'adressage en 2 moitiés
permet de vérifier très simplement si une adresse est protégée ou non
(par vérification du MSB de l'adresse)

Les fichiers \texttt{sys.bin} et \texttt{app.bin} contiennent déjà dans leurs
métadonnées les adresses où charger les segments qu'ils contiennent,
il n'y a donc pas à spécifier d'adresse dans l'appel au constructeur
du loader pour ce type de fichiers.

\newpage
tp2\_top.cpp : Segmentation de l'espace adressable
\lstinputlisting[language=c++, firstline=28, lastline=58]{tp2top.cpp}

tp2\_top.cpp : Paramétrage des caches 
\lstinputlisting[language=c++, firstline=81, lastline=87]{tp2top.cpp}

\section{Système d'exploitation : GIET}

\subsection{Question D1}

Pour effectuer un appel système, un utilisateur doit appeler la fonction
\texttt{sys\_call()} en donnant en arguments le numéro de l'apppel système en
question (quelle fonction appeler) et fournir les arguments en entrée
de cette fonction.

Les différents appels système prennent 4 arguments au maximum mais peuvent
en prendre moins, hors la fonction \texttt{sys\_call()} prend toujours 4 arguments.
Lorsqu'un appel système nécéssite moins de 4 arguments, il faut appeller
\texttt{sys\_call()} en donnant la valeur 0 pour les arguments non utilisés.

La fonction \texttt{syscall()} stocke le numéro de syscall et les arguments fournis
dans des registres. 

La transmission de ces informations (numéro et arguments de syscall)
se fait via les registres \texttt{v0} et \texttt{a0} à \texttt{a3}.

\subsection{Question D2}

\texttt{\_\_syscall\_vector[]} est un tableau de 32 mots contenant les adresses
des différentes fonctions appel système. C'est dans \texttt{sys\_handler.h} 
qu'est initialisé ce tableau ainsi que les macros permettant d'accéder à une
case par un nom intelligible plutôt que par un numéro brut.

\texttt{\_\_cause\_vector[]} fonctionne de la même manière sauf qu'il contient
l'ensemble des adresses des fonction de traitement des exceptions,
le tableau et les macros d'indexation sont définies dans \texttt{exc\_handler.h}.

\subsection{Question D3}

\textbf{Succession d'appels de fonctions :}
\begin{itemize}
\item \texttt{proctime()} appelle la fonction \texttt{sys\_call()} de \texttt{stdio.c}
\item \texttt{sys\_call()} execute l'instruction assembleur "syscall"
\item L'instruction assembleur syscall branche vers la fonction \texttt{\_sys\_handler} de \texttt{giet.s}
\item \texttt{\_sys\_handler,} va lire le tableau \texttt{\_\_syscall\_vector[]} pour obtenir
	l'adresse de l'appel système à exécuter, puis se branche à l'adresse obtenue, dans ce cas-ci
	se branche a \texttt{\_proctime()}.
\item \texttt{\_proctime()} de \texttt{drivers.c} est exécutée.
\texttt{\_sys\_handler} a branché.
\end{itemize}


\section{Génération du code binaire}

\subsection{Question E1}

Le code boot doit accéder aux registres protégés du processeur, or seuls
les programmes exécutés en mode superviseurs sont autorisés à y accéder
donc le code boot doit nécessairement être en mode superviseur.

\lstinputlisting[firstline=22, lastline=24]{reset.s}

\subsection{Question E2}

La convention permettant au code de boot de récupérer l'adresse du/des points
d'entrée dans le code applicatif est la suivante :
Au début du segment KDATA se trouve une table de sceaux, un tableau contenant
l'adresse de la/les fonctions qui sont des points d'entrée dans le code
applicatif.
Le code de boot n'a plus qu'à connaitre l'adresse de base du segment kdata
et il pourra y lire les adresse des points d'entrée dans le code applicatif.

Ici, il n'y a qu'une application utilisateur à lancer, donc qu'un point
d'entrée, qui se trouvera donc à l'adresse définie par la macro SEG\_KDATA\_BASE.

\lstinputlisting[firstline=12, lastline=30]{seg.ld}

\subsection{Question E3}

Si les adresses définies dans ces deux fichiers ne sont pas égales, cela peut
entrainer des bus error, des segmentation fault voire des écriture de données
vers les mauvais péripheriques.

Par exemple, si les adresses de base de la RAM et du TTY sont inversées
entre la définition logicielle et la définition matérielle, alors si un
programme veut écrire dans les registres du TTY, il enverra une commande
d'écriture à l'adresse de base du TTY (+ un potentiel offset), or pour le BCU,
cette adresse correspond a la RAM donc ce dernier selectionnera la RAM comme
cible, pour peu que l'OS soit charge en RAM vers les premières adresses,
celui-ci peut se retrouver corrpompu.
En fin de compte, un programme veut afficher un caractère à l'ecran et finit
par corrompre le système d'exploitation. 

\subsection{Question E4}

D'après le contenu de \texttt{sys.ld,} le segment RESET contient le code du fichier
reset.s et le segment KCODE contient le code du systeme d'exploitation,
en commençant par celui du fichier giet.s

\subsection{Question E5}

D'après le désassemblage de \texttt{sys.bin,} le segment RESET se situe entre les
adresses \texttt{0xbcf00000} et \texttt{0xbcf00024,}.
\texttt{0xbcf00000} est l'adresse du premier mot, et aussi l'adresse du premier
octet du segment. \texttt{0xbcf00024} est l'adresse du dernier mot, auquel
il faut ajouter $3$ pour obtenir l'adresse du dernier octet du segment.

Le segment est donc de taile :

$ (\text{adresse dernier octet}) - (\text{adresse premier octet}) + 1
  = (\texttt{0xbcf00024} + 3) - (\texttt{0xbcf00000}) + 1
  = 37 octets. $

De la même manière, le segment KCODE se situe entre \texttt{0x80000000}
et \texttt{0x80002224} donc est de taille $\texttt{0x2224} + 3 + 1 = 8744$ octets.

\subsection{Question E6}

\lstinputlisting[language=C, firstline=8, lastline=12]{main.c}

\subsection{Question E7}

La fonction système \texttt{\_tty\_read()} ne contient pas de boucle d'attente :
si aucun caractere n'est entré au clavier, la fonction n'attend pas et
renvoie 0.
En revanche, l'appel système \texttt{tty\_getc()} contient une boucle d'attente
maintenue tant que \texttt{\_tty\_read()} renvoie 0, autrement dit tant que
l'utilisateur n'a rien tapé au clavier.

C'est donc la fonction utilisateur et non pas la fonction noyau qui contient
la boucle d'attente.

\subsection{Question E8}

Le segment utilisateur CODE débute à l'adresse \texttt{0x00400000} et se termine à
\texttt{0x0040134f} (dernier octet), il a donc une taille de $\texttt{0x135f} + 1 = 4944$ octets.

\section{Exécution du code binaire sur le prototype virtuel}

\subsection{Question F1}

la première transaction sur le bus est une transaction rafale de lecture
de 4 mots a partir de l'adresse \texttt{0xbcf00000}
(adresse de base du segment RESET),
cela correspond à la lecture du bloc contenant la premiere instruction du
code de boot dans la ROM.

Au démarrage de la machine, la première chose que fait le processeur est
d'exécuter le code de boot mais comme au démarrage le cache est vide,
la lecture de la première instruction entraine un miss compulsif.
A cause de ce miss, le gestionnaire de cache va initier sur le bus
une transaction de lecture du bloc contenant cette instruction.

C'est au cycle 10 qu'est exécutée la première instruction du code de boot.
A ce cycle là, la réponse du cache d'instruction est valide et
l'instruction en question est \texttt{0x3c1d0200,} ce qui correspond bien au
code hexadecimal de l'instruction \texttt{lui sp, 0x200} qui est la première
instruction du code de boot (selon le fichier \texttt{sys.bin.txt}).

La deuxième transaction est encore une fois une transaction rafale de lecture
de 4 mots dans la ROM, mais cette fois à partir de l'adresse \texttt{0xbcf0000c},
4 mots plus loin que pour la première instruction.

Le processeur exécute le code de boot, la premiàre transaction a ramené un bloc
de 4 mots, contenant les 4 premières instructions du code de boot.
Celles-ci ont ensuite été exécutées, mais quand le SP passe à l'instruction 5,
il y a de nouveau un miss compulsif sur le cache d'instruction, qui va donc
entrainer une lecture du deuxieme bloc du segment RESET.
Ainsi de suite, il y aura miss compulsif toutes les 4 instructions.

\subsection{Question F2}

la première instruction du \texttt{main()} est l'instruction \texttt{0x27bdffd0}
(\texttt{addiu sp, sp, -48}) située à l'adresse \texttt{0x004012dc} d'après les binaires
decompiles.
Et d'après la trace, c'est à partir du cycle 57 que cette instruction
est exécutée.

\subsection{Question F3}

Une première transaction (commençant au cycle 90) lit un bloc contenant :
\begin{table}[H]
\centering
\begingroup
\setlength{\tabcolsep}{5pt}
\renewcommand{\arraystretch}{1.1}
\begin{tabular}{| l | l | l | l |}
\hline
Cycle	& Adresse	& Donnée Lue	& Traduction ASCII \\
\hline
94	& 0x01000078	& 6548200a	& He \\
\hline
95	& 0x0100007c	& 206f6c6c	& llo\_ \\
\hline
\end{tabular}
\endgroup
\end{table}


Une deuxieme transaction (commencant au cycle 114) lit le bloc suivant :
\begin{table}[H]
\centering
\begingroup
\setlength{\tabcolsep}{5pt}
\renewcommand{\arraystretch}{1.1}
\begin{tabular}{| l | l | l | l|}
\hline
Cycle	& Adresse	& Donnée Lue	& Traduction ASCII \\
\hline
116	& 0x01000080	& 0x6c726f57	& Worl \\
117	& 0x01000084	& 0x0a202164	& d! \\
\hline
\end{tabular}
\endgroup
\end{table}

Ainsi, c'est au cycle 94 que commence la première transaction corresponant
à la lecture de la chaine \texttt{"Hello world!"}.

%===============================================================================
%=========================  End of the Document  ===============================
%===============================================================================

\end{document}
