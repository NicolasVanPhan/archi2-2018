
\documentclass{article}

%% Packages for French writing
\usepackage[french]{babel}
\usepackage[utf8]{inputenc}
\usepackage[T1]{fontenc}
\usepackage{layout}

%% Packages for Math symbols
\usepackage[fleqn]{amsmath}
\usepackage{amssymb}
\usepackage{mathrsfs}

%% Packages for figures insertion
\usepackage{graphicx}
\usepackage{wrapfig}
\usepackage{framed}
\usepackage{float}

%% Package for document margin editing
\usepackage[top=2cm, bottom=2cm, left=2cm, right=2cm]{geometry}

%% Package for source code insertion
\usepackage{listings}
\usepackage{xcolor}
\definecolor{grey}{rgb}{0.97, 0.97, 0.97}
\definecolor{darkred}{rgb}{0.42, 0, 0}
\definecolor{darkblue}{rgb}{0, 0, 0.42}
\definecolor{darkgrey}{rgb}{0.22, 0.22, 0.82}
\definecolor{green}{HTML}{088A08}
\lstset{
  basicstyle=\small\sffamily\footnotesize,
  captionpos=b,
  numbers=left,
  numberstyle=\tiny,
  tabsize=4,
  frame=trBL,
  backgroundcolor=\color{grey},
  commentstyle=\color{green},
  keywordstyle=\color{darkblue}\bf,
  identifierstyle=\color{darkgrey},
  stringstyle=\color{darkred}
}

\setlength\parindent{0pt}
\setlength\parskip{3pt}
\title{ARCHI2 - Compte-rendu du TME3}
\author{Nicolas Phan}
\date{pour le 2 Mars 2018}
\begin{document}
\pagestyle{headings}
\maketitle
\tableofcontents
\newpage

%===============================================================================
%=========================  Introduction  ======================================
%===============================================================================

\section{Application logicielle}

\subsection{Question C1}

L'instruction \texttt{.word main} place le code commencant au label \texttt{main
} avant les tableaux A, B et C, le code de la fonction main commence
donc au début du segment \texttt{CODE}, autrement dit à l'adresse \texttt{0x00400000}.  

Le début de boucle commencce quatre instructions après le début de main,
donc 4 mots plus tard, donc $\text{@boucle} = \text{@main} + 4 \times 4 = \texttt{0x00400010}$

\subsection{Question C2}

Les directives \texttt{.word main } et \texttt{.space 124 } indiquent que le le tableau A se situe
$4 \text{(taille du word)} + 124$ octets plus loin que le début du segment DATA, donc :
\[ \text{@base 1} = \texttt{SEG\_DATA\_BASE} + 128 = \texttt{0x01000080} \]
Puis entre A et B il y a 20 mots plus un espace de 48 octets, d'où :
\[ \text{@base 2} = \text{@base 1} + 4 \times 20 + 48 = \texttt{0x01000100} \]
De la même manière :
\[ \text{@base 3} = \text{@base 2} + 4 \times 20 + 48 = \texttt{0x01000180} \]

\subsection{Question C3}

Placer \texttt{sw} après \texttt{bne} est une optimisation de code permettant de réduire le nombre
de cycles de gel par itérations.
Apres le \texttt{bne}  l y aura un Delayed Slot, autant de pas y mettre un \texttt{Nop}  mais
une instruction utile.

\subsection{Question C4}

Ici :
\begin{align*}
\text{nombre cycles par iteration}  &= \text{nb instructions} + \text{nb cycles gel} \\
&= 7 + 0 \\
&= 7 \text{cycles/it} \\
\end{align*}


\section{Fonctionnement du cache instruction}

\subsection{Question D1}

Les lignes de cache mesurent $4$ mots = $16$ octets = $2^4$ octets
donc le champ \texttt{BYTE}  comporte $4$ bits.

Une ligne de cache mesure $16$ octets et le cache est de $128$ octets,
il comporte donc $ \frac{128}{16} = 8$ cases.
Le cache comporte $8$ cases donc $8$ ensemble (direct mapping)
donc le champ \texttt{SET}  comporte $3$ bits. 

Par soustraction, le champ \texttt{TAG}  comporte $32 - 4 - 3 = 25$ bits

Pour l'adresse \texttt{0x00400000} , nous avons :

\texttt{BYTE = 0} 

\texttt{SET = 0} 

\texttt{TAG = 0000 0000 0100 0000 0000 0000 0 = 0x8000}

\subsection{Question D2}

La toute première ligne lui produit un miss compulsif,
elle correspond à l'adresse \texttt{0x00400000}  ($\texttt{SET}  = 0$, $\texttt{TAG}  = \texttt{0x8000} $)
donc le controlleur de cache ramène la ligne de cache comportant
les 4 premières instructions dans la case 0 du cache.

Les 4 premières instructions sont exécutées puis la lecture de la 5ème
produit un miss compulsif, cette instruction à l'adresse \texttt{0x00400010}  
($\texttt{SET}  = 1$, $\texttt{TAG}  = \texttt{0x8000} $) donc la ligne de cache
correspondante sera ramenée à la case 1.

Par cette logique de quatre en quatre, ce seront donc les instructions
\texttt{lui} , \texttt{lw \$10} et \texttt{add}  qui entraineront un miss sur le cache d'instructions.

Etat du cache d'instructions à la fin de la première itération :
\begin{table}[H]
\centering
\begingroup
\setlength{\tabcolsep}{5pt}
\renewcommand{\arraystretch}{1.1}
\begin{tabular}{| l | l | l | l | l | l | l |}

\hline
CASE & TAG & V & WORD3 & WORD2 & WORD1 & WORD0 \\
\hline
0 &\texttt{0x8000} & 1 & \texttt{li} & \texttt{li} & \texttt{addiu} & \texttt{lui} \\
\hline
1 &\texttt{0x8000} & 1 & \texttt{addi} & \texttt{addi} & \texttt{lw} & \texttt{lw} \\
\hline
2 & \texttt{0x8000} & 1 & \texttt{lui} & \texttt{sw} & \texttt{bne} & \texttt{add} \\
\hline
3&&0&&&&\\
\hline
4&&0&&&&\\
\hline
5&&0&&&&\\
\hline
6&&0&&&&\\
\hline
7&&0&&&&\\
\hline

\end{tabular}
\caption{Etat du cache d'instructions}
\endgroup
\end{table}


\subsection{Question D3}

L'exécutions des itérations suivantes ne changent pas l'état du cache
d'instructions car toutes les instructions à exécuter sont déjà présentes
dans le cache d'instructions.
A la fin de la 20ème itération, l'état du cache est le même qu'à la fin
de la première.

L'exécution de la boucle entraine donc 2 miss (pour la première itération)
pour l'exécution de $20 \times 2 = 40$ instructions de lecture. Le taux de miss est donc
environ de $\frac{2}{40} = 5$ \%.

\subsection{Question D4}

L'état \texttt{MISS\_SELECT}  est indispensable pour les caches de niveau d'associativité
2 où plus. Dans ces caches, plusieurs cases de cache sont possibles pour
stocker une ligne de cache donnée, donc l'automate doit nécessairement choisir
la case où stocker une ligne.

\subsection{Question D5}


\begin{table}[H]
\centering
\begingroup
\setlength{\tabcolsep}{5pt}
\renewcommand{\arraystretch}{1.1}
\begin{tabular}{| l | p{14cm} |}

\hline
\textbf{Etat} & \textbf{Signification} \\
\hline
\hline
IDLE
&
L'automate n'a affaire à aucun MISS
\\
\hline

ERROR
&
Une adresse invalide a été demandée
\\
\hline

MISS\_SELECT
&
Un miss cachable s'est produit, l'automate sélectionne la case où
stocker la ligne à ramener.
\\
\hline

MISS\_WAIT
&
Un miss cachable s'est produit, l'automate attend que l'automate PIBUS\_FSM
ait amenée la ligne de cache dans le registre tampon.
\\
\hline

MISS\_UPDT
&
L'automate envoie la ligne récupérée dans le registre tampon vers la case
de cache choisie.
\\
\hline

UNC\_WAIT
&
Un miss non-cachable s'est produit, l'automate attend que PIBUS\_FSM ait
amenée la donnée demandée dans le registre tampon.
\\
\hline

UNC\_GO
&
Un miss non-cachable s'est produit, l'automate redirige la donnée recue
dans le registre tampon vers le processeur.
\\
\hline

\end{tabular}
\caption{Signification des états}
\endgroup
\end{table}


\begin{table}[H]
\centering
\begingroup
\setlength{\tabcolsep}{5pt}
\renewcommand{\arraystretch}{1.1}
\begin{tabular}{| l | l | p{12cm} |}

\hline
Transition & Expression booleenne & Signification \\
\hline
\hline
C
&
$\overline{\texttt{IREQ}}$ + $\overline{\texttt{IMISS}}$
&
Si le processeur ne demande rien ou s'il demande quelque chose qui est bien
dans le cache, alors il n'y a pas miss.
\\
\hline

A
&
\texttt{IREQ}  . \texttt{IMISS}  . \texttt{IUNC}  
&
Le processeur demande une donnée non cachable, cela a (nécéssairement) entrainé
un miss.
\\
\hline

B
&
\texttt{IREQ}  . \texttt{IMISS}  . $\overline{\texttt{IUNC}}$
&
Le processeur demande une donnée cachable mais celle-ci n'est pas dans le cache.
\\
\hline
\hline
I
&
1
&
\\
\hline

F
&
\texttt{VALID}  + $\overline{\texttt{ERROR}}$
&
L'automate a recu une réponse sans erreur de la part de PIBUS\_FSM,
il passe à la mise à jour du cache.
\\
\hline

G
&
\texttt{VALID}  +\texttt{ERROR}  
&
L'automate a recu une réponse d'erreur de la part de PIBUS\_FSM,
il passe à l'état d'erreur.
\\
\hline

H
&
 $\overline{\texttt{VALID}}$
&
PIBUS\_FSM n'a toujours pas donné sa réponse, l'automate continue d'attendre.
\\
\hline

N
&
1
&
\\

\hline
\hline

J
&
 $\overline{\texttt{VALID}}$
&
PIBUS\_FSM n'a toujours pas donné sa réponse, l'automate continue d'attendre.
\\
\hline

K
&
\texttt{VALID}  +\texttt{ERROR}  
&
L'automate a recu une réponse d'erreur de la part de PIBUS\_FSM,
il passe à l'état d'erreur.
\\
\hline

L
&
\texttt{VALID}  + $\overline{\texttt{ERROR}}$
&
L'automate a recu une réponse sans erreur de la part de PIBUS\_FSM,
il passe à l'état de redirection de la donnée vers le processeur.
\\
\hline

M
&
1
&
\\
\hline

\end{tabular}
\caption{Transitions de l'automate}
\endgroup
\end{table}

\subsection{Question D6}

L'activation du signal RESETN force cet automate anisi que les deux autres
à l'état IDLE. De plus, l'activation du signal RESETN entraine l'invalidation
de 100% des caches.


\section{Fonctionnement du cache de donnees}

\subsection{Question E1}

Pour l'adresse \texttt{0x01000080} (\texttt{A[0]}), nous avons :

\texttt{BYTE = 0} 

\texttt{SET = 0} 

\texttt{TAG = 0000 0001 0000 0000 0000 0000 1 = 0x20001}

Pour l'adresse \texttt{0x01000100} (\texttt{B[0]}), nous avons :

\texttt{BYTE = 0} 

\texttt{SET = 0} 

\texttt{TAG = 0000 0001 0000 0000 0000 0001 0 = 0x20002}

Le premier \texttt{lw} entraine un miss compulsif où les 4 premieres cases
du tableau A sont ramenées dans la case 0 du cache. Puis le deuxième
\texttt{lw} entraine aussi un miss compulsif, sauf que la case associée
aux 4 premiers mots du tableau B est aussi la case 0, donc le début du
tableau A est évincé du cache.

Etat du cache de données à la fin de la première itération :
\begin{table}[H]
\centering
\begingroup
\setlength{\tabcolsep}{5pt}
\renewcommand{\arraystretch}{1.1}
\begin{tabular}{| l | l | l | l | l | l | l |}

\hline
CASE & TAG & V & WORD3 & WORD2 & WORD1 & WORD0 \\
\hline
0 &\texttt{0x20002} & 1 & 104 & 103 & 102 & 101 \\ 
\hline
1&&0&&&&\\
\hline
2&&0&&&&\\
\hline
3&&0&&&&\\
\hline
4&&0&&&&\\
\hline
5&&0&&&&\\
\hline
6&&0&&&&\\
\hline
7&&0&&&&\\
\hline

\end{tabular}
\caption{Etat du cache de données}
\endgroup
\end{table}

\subsection{Question E2}

A la deuxième itération, le premier lw entrainera un miss conflit cette fois,
car il a été évincé précédemment par le tableau B : Les deux tableaux sont
en conflit donc lorsque les lw n'entrainent pas de miss compulsifs, ils
entrainent toujours des miss conflit. Le taux de miss est donc de 100 \%.

Les miss seront compulsifs aux itétations $n$ où $n \mod 4 = 1$, là où la case du
tableau demandée est au début d'une nouvelle ligne.
Dans les autres cas il s'agira de miss conflits.

Etat du cache de données à la fin de la dernière itération :
\begin{table}[H]
\centering
\begingroup
\setlength{\tabcolsep}{5pt}
\renewcommand{\arraystretch}{1.1}
\begin{tabular}{| l | l | l | l | l | l | l |}

\hline
CASE & TAG & V & WORD3 & WORD2 & WORD1 & WORD0 \\
\hline
0 &\texttt{0x20002} & 1 & 104 & 103 & 102 & 101 \\ 
\hline
1 &\texttt{0x20002} & 1 & 108 & 107 & 106 & 105 \\ 
\hline
2 &\texttt{0x20002} & 1 & 112 & 111 & 110 & 109 \\ 
\hline
3 &\texttt{0x20002} & 1 & 116 & 115 & 114 & 113 \\ 
\hline
4 &\texttt{0x20002} & 1 & 120 & 119 & 118 & 117 \\ 
\hline
5&&0&&&& \\
\hline
6&&0&&&& \\
\hline
7&&0&&&& \\
\hline
\end{tabular}
\caption{Etat du cache de données}
\endgroup
\end{table}


\subsection{Question E3}

\begin{table}[H]
\centering
\begingroup
\setlength{\tabcolsep}{5pt}
\renewcommand{\arraystretch}{1.1}
\begin{tabular}{| l | l | p{12cm} |}

\hline
Transition & Expression booleenne & Signification \\
\hline
\hline
C
&
$\overline{\texttt{DREQ}}$ + ($\overline{\texttt{WRITE}}$ . $\overline{\texttt{DMISS}}$)
&
Si le processeur ne demande rien ou s'il demande à lire quelque chose qui est bien
dans le cache, alors il n'y a pas miss.
\\
\hline

A
&
\texttt{DREQ} . $\overline{\texttt{WRITE}}$ . \texttt{DMISS}  . \texttt{DUNC}  
&
Le processeur demande à lire une donnée non cachable, cela a (nécéssairement) entrainé
un miss.
\\
\hline

B
&
\texttt{DREQ} . $\overline{\texttt{WRITE}}$ . \texttt{DMISS} . $\overline{\texttt{DUNC}}$
&
Le processeur demande à lire une donnée cachable mais celle-ci n'est pas dans le cache.
\\

\hline
\hline
D
&
\texttt{DREQ}  . \texttt{WRITE}  . $\overline{\texttt{DMISS}}$
&
Le processeur demande à écrire à une adresse présente dans le cache, l'automate
passe à l'état de mise à jour du cache
\\
\hline

E
&
\texttt{DREQ}  . \texttt{WRITE}  . \texttt{DMISS}
&
Le processeur demande à écrire à une adresse absente du cache, l'automate
n'a pas à mettre à jour le cache, il passe directement à l'état où il poste
cette requete dans le TEP. 
\\

\hline
P
&
1
&
L'automate a fini de mettre à jour le cache suite à une demande d'écriture,
il passe donc à l'état où il poste cette requête dans le TEP.
\\


\hline
\hline
I
&
1
&
\\
\hline

F
&
\texttt{VALID}  + $\overline{\texttt{ERROR}}$
&
L'automate a recu une réponse sans erreur de la part de PIBUS\_FSM,
il passe à la mise à jour du cache.
\\
\hline

G
&
\texttt{VALID}  +\texttt{ERROR}  
&
L'automate a recu une réponse d'erreur de la part de PIBUS\_FSM,
il passe à l'état d'erreur.
\\
\hline

H
&
 $\overline{\texttt{VALID}}$
&
PIBUS\_FSM n'a toujours pas donné sa réponse, l'automate continue d'attendre.
\\
\hline

N
&
1
&
\\

\hline
\hline

J
&
 $\overline{\texttt{VALID}}$
&
PIBUS\_FSM n'a toujours pas donné sa réponse, l'automate continue d'attendre.
\\
\hline

K
&
\texttt{VALID}  +\texttt{ERROR}  
&
L'automate a recu une réponse d'erreur de la part de PIBUS\_FSM,
il passe à l'état d'erreur.
\\
\hline

L
&
\texttt{VALID}  + $\overline{\texttt{ERROR}}$
&
L'automate a recu une réponse sans erreur de la part de PIBUS\_FSM,
il passe à l'état de redirection de la donnée vers le processeur.
\\
\hline

M
&
1
&
\\
\hline

\end{tabular}
\caption{Transitions de l'automate}
\endgroup
\end{table}

\subsection{Question E4}

La différence entre les transition de sortie de IDLE et WRITE\_REQ est la
condition de rebouclage vers le même état. A l'état WRITEREQ, l'automate reste
dans ce même état tant que le buffer d'écriture est plein. En effet, l'automate
a une écriture a poster mais le TEP est pleine, et à cause de la politique
write-through, cette écriture doit être répercutée immédiatement dans la RAM,
donc la requête d'écriture doit absolument être postée avant que le processeur
passe à l'exécutions des instructions suivantes. L'automate reste donc dans
l'état WRITEREQ et le processeur gèle jusqu'à ce qu'une place se libère dans
le TEP.


\section{Accès au PIBUS}


\subsection{Question F1}

Supposons que le processeur fasse une demande de lecture alors qu'il y a des
requêtes d'écriture dans le TEP. Pour peu que la lecture demandée concerne la
même adresse qu'une écriture postée, répondre à la demande de lecture en premier
entrainerait une inconsistence mémoire, la donnée renvoyée au processeur serait
obsolète. Une manière de remédier à cela est de comparer les adresses des
requêtes d'écritures de la TEP avec l'adresse de la lecture demandée, mais cela
apporte des complication en terme de conception.

Une autre solution, plus
simple mais plus radicale, consiste à toujours effectuer toutes les écritures
dans le TEP avant d'effectuer quelconque lecture. De cette manière, nous sommes
assurés que la consistence mémoire est respectée, toutes les données renvoyées
au processeur seront bien à jour. C'est la solution retenue ici, les écritures
sont donc toujours prioritaires.

L'inconvénient est que si le TEP est rempli, une demande de lecture entrainera
beaucoup de cycles de gel car la demande sera satisfaite qu'une fois que TOUTES
les écritures du TEP auront été effectuées, et tant que la lecture n'est pas
terminée le processeur reste gelé. Le processeur gèlera donc le temps
d'effectuer les N écritures postées plus la lecture en question.

\subsection{Question F2}

Le mécanisme utlisé par les automates clients pour transmettre une requête
à l'automate serveur est le mécanisme de tampon protégé par une bascule RS,
c'est le mécanisme le plus simple et le plus utilisé pour la communication
entre automates.

Ce mécanisme consiste en un registre tampon, ici ADDR de 28 bits
(et non pas 32 car les adresses fournies sont nécéssairement alignées en
mémoire, et étant donné que les lignes de caches font $2^4$ octets, les 4 LSB
des adresses sont nuls), c'est l'adresse fournie par le client au serveur.
La deuxième partie du mécanisme est la bascule RS (ici REQ) qui est un automate
à deux états indiquant si une requête est en suspens et dans le même temps,
détermine qui a la parole (client ou serveur).

Lorsque REQ est à 0, aucune requête n'est en suspens, le client a la parole et
la prendra lorsqu'il aura une requête à soumettre, auquel cas il peut mettre
REQ à 1. Le serveur, quant à lui, ne peut pas modifier REQ.

Lorsque REQ est à 1, une requête est en cours de traitement, le serveur a la
parole et la prendra lorsqu'il aura satisfait la requête, auquel cas il mettra
REQ à 0. Le client, quand à lui, ne peut pas modifier REQ.


Pour transmettre la réponse du serveur aux clients, le même mécanisme est
employé. Le signal en sortie de la bascule RS est IRSP et vaut 1 lorsqu'une
réponse du serveur est disponible et attent d'être récupérée par le client,
c'est au client de remettre IRSP à 0 lorsqu'il récupère la réponse.
IRSP vaut 0 lorsqu'aucune réponse n'est disponible, c'est au serveur de mettre
à 1 lorsqu'il en délivrera une.

\subsection{Question F3}

Les automates ICACHEFSM et DCACHEFSM n'ont pas besoin de savoir qu'une écriture
s'est terminée car s'ils veulent effectuer une lecture, leurs requêtes au
serveur resteront en attente tant que toutes les écritures n'auront pas été
effectuées.

F4]

\begin{table}[H]
\centering
\begingroup
\setlength{\tabcolsep}{5pt}
\renewcommand{\arraystretch}{1.1}
\begin{tabular}{| l | l |}

\hline
Transision & Expression booleenne \\

\hline
Z
&
\texttt{
ROK . $\overline{\texttt{SC}}$ . $\overline{\texttt{IUNC}}$
. $\overline{\texttt{IMISS}}$ . $\overline{\texttt{DUNC}}$ . $\overline{\texttt{DMISS}}$
} \\

\hline


X
&
\texttt{
$\overline{\texttt{ROK}}$ + SC
} \\
\hline

B'
&
\texttt{
$\overline{\texttt{GNT}}$
} \\
\hline

B
&
\texttt{
GNT
} \\
\hline

C
&
\texttt{
1
} \\
\hline

D'
&
\texttt{
$\overline{\texttt{WAIT}}$
} \\
\hline

D
&
\texttt{
WAIT
} \\
\hline


Y
&
\texttt{
ROK . $\overline{\texttt{SC}}$ . (IUNC + IMISS + DUNC + DMISS)
} \\
\hline

E'
&
\texttt{
$\overline{\texttt{GNT}}$
} \\
\hline

E
&
\texttt{
GNT
} \\
\hline

F
&
\texttt{
LAST
} \\
\hline

F'
&
\texttt{
$\overline{\texttt{LAST}}$
} \\
\hline

G
&
\texttt{
LAST
} \\
\hline

G'
&
\texttt{
$\overline{\texttt{LAST}}$
} \\
\hline

H'
&
\texttt{
$\overline{\texttt{WAIT}}$
} \\
\hline

H
&
\texttt{
WAIT
} \\
\hline

\end{tabular}
\caption{Transitions de l'automate}
\endgroup
\end{table}


\section{Expérimentation par simulation}

\subsection{Question G1}

Au cycle 10, le processeur recoit via l'interface IRSP la première instruction
à exécuter (l'instruction lui sp 0x200 a l'adresse 0xbcf00000, c'est la
première instruction du code de boot au debut du segment RESET).
C'est au cycle suivant qu'il commence a l'exécuter.

Au cycle 0 le processeur demande l'instruction a 0xbcf00000 et le cache
d'instructions fait miss, cette instruction sera disponible via l'interface IRSP
qu'à la fin du cycle 10 alors qu'elle aurait été disponible a la fin du cycle 0
s'il y avait eu hit. Il y a donc 10 cycles de gel (cycles 1 a 10) causés par
le miss compulsif.

\subsection{Question G2}

Au cycle 47, le processeur demande la première instruction du main
(à l'adresse 0x00400000),
le cache fait miss. C'est au cycle 57 que cette instruction commence à être
exécutée,
nous retrouvons bien les 10 cycles d'attentes calcules precedemment.

\subsection{Question G3}

Les instructions lw mettent entrainent chacune 12 cycles de gel à cause des
MISS qu'elles entrainent.

\subsection{Question G4}

Pour les itérations autres que la première, le coût du miss du premier lw
est plus important car au moment du miss, il y a une requête postée dans le 
TEP (venant du sw de l'itération précédente) et le controlleur de cache
effectue d'abord toutes les écritures dans le TEP avant d'effectuer une lecture.
Le miss du premier lw entraine donc des cycles de gel le temps d'effectuer
une écriture en plus d'une lecture.



%===============================================================================
%=========================  End of the Document  ===============================
%===============================================================================

\end{document}

